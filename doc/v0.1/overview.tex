\section{Découpage du projet}
\begin{sidewaysfigure}[ftbp]
    \includegraphics[width=0.9\textwidth]{imgs/overview}
    \caption{Schéma logique du projet}
\end{sidewaysfigure}

Le projet peut être découpé en 5 grosses parties :
\subsection{Interface utilisateur}

Cette partie s'occupe des différents moyen d'interagir avec le shell. Elle doit
être capable de lire des commandes depuis un fichier, un argument ou depuis un
mode interactif. Ce dernier est le plus important car il intéragit directement
avec l'utilisateur, c'est pourquoi nous apporterons un soin tout particulier au
confort d'utilisation (historique, déplacement, complétion,~\dots)

\subsection{Lexer, parser, AST}

Le shell doit essayer de comprendre ce que veux faire l'utilisateur. En se
basant sur une grammaire LL(1), le shell doit construire une suite d'actions
logiques sous la forme d'un arbre (AST). Les entrées provenant d'un humain, il
doit être assez robuste pour résister aux multiples erreurs : « never trust
user input ».

\subsection{Gestions des options du shell}

Cette partie permet de définir les réactions du shell vis-à-vis de différentes
actions. L'utilisateur peut donc changer le comportement du globbing ou de
l'extension à travers des arguments qui seront gérés ici.

\subsection{Évaluation de l'AST}

Il s'agit ici d'interpréter l'AST que le parser a produit pour effectuer les
différentes actions (structures de contrôles, commandes,~\dots). Tout en gérant
l'expansions des différentes différentes composantes (variables,
globbing,~\dots).

\subsection{Builtins, commandes, gestion des processus}

Il faut coder certaines commandes du shell appellées « builtins ». Si
l'utilisateur utilise une commande externe, cette partie s'assurera de trouver
cette commande et de suivre son activité (code de retour, redirections,~\dots).
