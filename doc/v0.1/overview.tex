\section{Découpage du projet}

Le projet peut être découpé en 5 grosses parties :
\subsection{Interface utilisateur}

Cette partie s'occupe des différents moyen d'interagir avec le shell. Elle doit
être capable de lire des commandes depuis un fichier, un argument ou depuis un
mode interactif. Ce dernier est le plus important car il intéragit directement
avec l'utilisateur, c'est pourquoi nous apporterons un soin tout particulier au
confort d'utilisation (historique, déplacement, complétion,~\dots)

Afin de donner la possibilité à l'utilisateur de manipuler la ligne de commande,
le module readline devra être capable d'interprêter les combinaisons de touche
telles que « flèche gauche », « alt b », « contrôl w », etc, et de reproduire le
comportement attendu (déplacement du curseur, effacement de caractères). Nous
utiliserons pour cela les fonctions de l'API termcap et les terminfos.

Pour proposer des fonctionnalités avancées et pratiques pour l'utilisateur, le
module readline devra communiquer avec les différentes parties du projet et avec
l'environement dans lequel 42sh aura été appelé. L'auto complétion par exemple,
dans sa forme basique, listera les fichiers disponibles dans un dossier pour
proposer une completion de nom de fichier, mais elle pourrait par la suite
communiquer avec le parseur pour connaitre le prochain type de token attendu
et proposer l'auto completion des commandes ou des mot clefs. Elle pourrait
encore demander à l'execution d'étendre le mot courant si le parseur le détecte
comme étant un token valide à étendre.

\subsection{Lexer, parser, AST}

Le shell doit essayer de comprendre ce que veux faire l'utilisateur. En se
basant sur une grammaire LL(1), le shell doit construire une suite d'actions
logiques sous la forme d'un arbre (AST). Les entrées provenant d'un humain, il
doit être assez robuste pour résister aux multiples erreurs : « never trust
user input ».

\subsection{Gestions des options du shell}

Cette partie permet de définir les réactions du shell vis-à-vis de différentes
actions. L'utilisateur peut donc changer le comportement du globbing ou de
l'extension à travers des arguments qui seront gérés ici.

\subsection{Évaluation de l'AST}

Il s'agit ici d'interpréter l'AST que le parser a produit pour effectuer les
différentes actions (structures de contrôles, commandes,~\dots). Tout en gérant
l'expansions des différentes différentes composantes (variables,
globbing,~\dots).

\subsection{Builtins, commandes, gestion des processus}

Il faut coder certaines commandes du shell appellées « builtins ». Si
l'utilisateur utilise une commande externe, cette partie s'assurera de trouver
cette commande et de suivre son activité (code de retour, redirections,~\dots).

\begin{sidewaysfigure}[H]
    \includegraphics[width=0.9\textwidth]{imgs/overview}
    \caption{Schéma logique du projet}
\end{sidewaysfigure}

\begin{center}
    \includegraphics[width=0.7\textwidth]{imgs/topdown}
\end{center}
