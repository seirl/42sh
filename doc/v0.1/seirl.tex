\subsection{\texttt{pietri\_a} - Antoine \textsc{Pietri}}

\begin{itemize}
    \item Interprétation abstraite de l'AST
    \item Expansion des variables
\end{itemize}
