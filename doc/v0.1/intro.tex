\section{Aperçu du project}

\subsection{42sh}

42sh est un shell compatible POSIX écrit en C99.

\subsection{Build system}

Nous utiliserons CMake comme \emph{build system}. Les raisons de ce choix
sont les suivantes :
\begin{itemize}
    \item Moderne
    \item Propre
    \item Configuration relativement facile
\end{itemize}

\subsection{Suite de tests}

La suite de tests sera écrite en python3.3. Elle étendra le module
\texttt{unittest} de la bibliothèque standard pour l'adapter à nos besoins :

\begin{itemize}
    \item charger des tests depuis des fichiers ;
    \item sélectionner la catégorie de tests à effectuer ;
    \item afficher les résultats des tests pour chaque catégorie.
\end{itemize}

\begin{listing}[H]
    \begin{minted}[linenos,
                   frame=lines,]{bash}
        42sh$ make check
        cd check && ./run_tests.py

        Category: basic

        basic/00_hello ... ok
        basic/01_hello ... ok

        Success: 100%

        Final results:
        Success: 100%
    \end{minted}
    \caption{Exemple de sortie de la suite de tests.}
\end{listing}

\subsection{Communication}

Les développeurs utiliseront IRC pour les discutions quotidiennes.

\subsection{Intégration continue}

Chaque commit déclenchera un système charger de vérifier l'absence de
régressions.
\begin{itemize}
    \item Compilation avec gcc et clang
    \item Suite de tests
    \item Analyse valgrind (fd + mémoire)
\end{itemize}

Le projet sera automatiquement compilé sur différents systèmes
(variations d'architecture, système d'exploitation, libc, etc.). Puis testé
par le système décrit précedemment.

Actuellement les systèmes suivants sont disponibles :
\begin{itemize}
    \item Archlinux
    \item FreeBSD 9.1
\end{itemize}

Un bot IRC annoncera chaque commit et le résultat des tests.
