\section{Aperçu du project}

\subsection{42sh}

42sh est un shell compatible POSIX écrit en C99.

\subsection{Build system}

Nous utiliserons CMake comme \emph{build system}.

\subsection{Suite de tests}

La suite de tests sera écrite en python3.3. Elle étendra le module
\texttt{unittest} de la bibliothèque standard pour l'adapter à nos besoins :

\begin{itemize}
    \item charger des tests depuis des fichiers ;
    \item sélectionner la catégorie de tests à effectuer ;
    \item afficher les résultats des tests pour chaque catégorie.
\end{itemize}

\begin{listing}[H]
    \begin{minted}[linenos,
                   frame=lines,]{bash}
        42sh$ make check
        cd check && ./run_tests.py

        Category: hello

        hello/00_hello ... ok
        hello/01_hello ... ok

        Ran: 2 Failure: 0 Errors: 0 Rate: 100%

        Category: basic

        basic/00_hello ... ok
        basic/01_hello ... ok

        Ran: 2 Failure: 0 Errors: 0 Rate: 100%
    \end{minted}
    \caption{Exemple de sortie de la suite de tests.}
\end{listing}

\subsection{Communication}

Les développeurs utiliseront IRC pour les discutions quotidiennes.

\subsection{Intégration continue}

Chaque commit déclenchera une compilation du projet sur différents systèmes
(variations d'architecture, système d'exploitation, libc, etc.) et la suite de
tests sera lancée.

Un bot IRC annoncera chaque commit et le résultat des tests.
